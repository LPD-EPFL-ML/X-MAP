\begin{table}[t]
\centering
\setlength\extrarowheight{0.8pt}
\begin{tabular}{ |c|p{6.5cm}| }
  \hline
  \multicolumn{2}{|c|}{Notations} \\
  \hline
  $\mathcal{D}$ & {\small any single domain} \\ \hline
  $\mathcal{U}$ & {\small the set of users in a domain} \\ \hline
  $\mathcal{I}$ & {\small the set of items in a domain} \\ \hline
  $r_{u,i}$ & {\small the rating provided by a user $u$ for an item $i$} \\ \hline
  $M_D$ & {\small the matrix for the ratings provided by the users\footnotemark} \\ \hline
  $\bar{r}_u$ & {\small the average rating of $u$ over all items rated by $u$} \\ \hline
  $\bar{r}_i$ & {\small the average rating for $i$ over all users who rated $i$} \\ \hline
  $X_u$  & {\small the set of items rated by $u$} \\ \hline
  $Y_i$ & {\small the set of users who rated $i$} \\ \hline
  $\tau(u,v)$ & {\small the Pearson correlation between $u$ and $v$}\\
  \hline
\end{tabular}
\caption{ Notations}
\label{fig:NotationTable}
\end{table}

\footnotetext{ Normally, $M_D$ is a sparse matrix. Hence, if $u$ has not rated $i$, then we use the average rating of $i$ as $r_{u,i}$ to complete $M_D$.}

