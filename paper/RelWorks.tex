\vspace{-2mm}
\section{Related Work}
\label{RelWorks}

\noindent{\bf Heterogeneous trends.}
Research on heterogeneous personalization is relatively new. There has been however a few approaches to tackle the problem which we discuss below.
% along with their drawbacks for a real-world deployment. 

%We summarize their characteristics in Table~\ref{fig:relworks}. 
%The main research trends in heterogeneous recommendation were identified by Loizou in his doctoral thesis~\cite{loizou2009recommend} which are as follows.

{\it Smart User Models.} Gonz{\'a}lez et. al introduced the notion of Smart User Models (SUMs)~\cite{gonzalez2005multi}. The idea is to aggregate heterogeneous information to build user profiles that are applicable across different domains. SUMs relies on users' emotional context which are however difficult to capture. Additionally, it has been shown that users' ratings vary frequently with time depending on their emotions~\cite{amatriain2009like}.

{\it Web Monitoring.} 
% Web user agents construct user profiles by monitoring their web activity across different domains. 
Hyung et. al designed a web agent which profiles user interests across multiple domains and leverage this information for personalized web support~\cite{kook2005profiling}. Tuffield et. al proposed Semantic Logger, a meta-data acquisition web agent that collects and stores any information (from emails, URLs, tags) accessed by the users~\cite{tuffield2006semantic}. However, web agents are considered a threat to users' privacy as users' data over different e-commerce applications are stored in a central database administered by the web agent.

{\it Cross-domain Mediation.} Berkovsky et. al~\cite{berkovsky2007cross} proposed the idea of cross-domain mediation to compute recommendations by aggregating data over several recommenders. We showed empirically that \crossrec  outperforms cross-domain mediation in Figures~\ref{fig:varyK} and \ref{fig:varyTrain}.

In contrast, \crossrec introduces a new trend in heterogeneous personalization where the user profile from a source domain is leveraged to generate an \emph{artificial} private or non-private AlterEgo profile in a target domain. The AlterEgo profiles can even be exchanged between e-commerce companies like Netflix, Last.fm due to the privacy guarantee in \crossrec.

\noindent{\bf Merging preferences.} One could also view the 
problem as that of merging single-domain user preferences. Through this angle, several approaches can be considered.

{\it Rating aggregation.} This approach is based on aggregating user ratings over several domains into a single multi-domain rating matrix~\cite{berkovsky2007distributed,winoto2008if, berkovsky2007cross}. Berkovsky et. al showed that this approach can tackle cold-start problems in collaborative filtering~\cite{berkovsky2007distributed}. We showed empirically that \crossrec easily outperforms such rating aggregation based approaches~\cite{berkovsky2007cross}.

{\it Common representation.} This approach is based on a common representation of user preferences from multiple domains either in the form of a \emph{social tag}~\cite{szomszor2008correlating,szomszor2008semantic} or \emph{semantic relationships between domains}~\cite{loizou2009recommend}. Shi et. al developed a Tag-induced  Cross-Domain Collaborative Filtering (TagCDCF) to overcome cold-start problems in collaborative filtering~\cite{shi2011tags}. TagCDCF exploits shared tags to link different domains. They thus need additional tags to bridge the domains. \crossrec can bridge the domains based on the ratings provided by users using its novel \graphsim measure without requiring any such additional information which might be difficult to collect in practice.

{\it Linked preferences.} This approach is based on linking user preferences in several domains~\cite{nakatsuji2010recommendations, cremonesi2011cross}. We showed empirically that \crossrec  outperforms such linked preference based approaches~\cite{cremonesi2011cross} in Figures~\ref{fig:varyK} and \ref{fig:varyTrain}.

{\it Domain-independent features.} This approach is based on mapping user preferences to domain-independent features like \emph{personality features}~\cite{cantador2013relating} or \emph{user-item interaction features}~\cite{loni2014cross}. This approach again requires additional information like \emph{personality scores} which might not be available for all users. 


%\crossrec improves the rating aggregation and linked preferences approaches and uses them along with the \emph{AlterEgo representation} to provide improved heterogeneous recommendations.













\iffalse
\begin{table*}
\begin{center}
\begin{tabular}{ |c|c|c| }
   \hline
 {\bf User Overlap} & {\bf Item overlap} & {\bf User-Item overlap} \\
   \hline
  

{\bf \crossrec} & Stewart et al. 2009~\cite{stewart2009cross} &Berkovsky et al. 2007~\cite{berkovsky2007distributed} \\
  Loni et al. 2014~\cite{loni2014cross} &- & Tiroshi et al. 2012~\cite{tiroshi2012domain} \\


    \hline
\end{tabular}
\end{center}
\vspace{-4mm}
\caption{\bf Related works based on overlap.}
  \label{fig:relworks}
\vspace{-3mm}
\end{table*}
\fi